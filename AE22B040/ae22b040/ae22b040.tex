\documentclass{article}
\usepackage{amsmath} % Required for inserting images

\title{Biot Savart Law-Equation of my choice}
\author{Padma priya R }
\date{June 2023}

\begin{document}
\maketitle
\section{AE22B040}
The vector form of biot savart law is 
\begin{align*}
    \mathbf{B}(\mathbf{r}) =\frac{\mu_0}{4\pi} \int \frac{I\, d\mathbf{l} \times \mathbf{r'}}{|\mathbf{r'}|^3}
\end{align*}
where
\begin{itemize}
  \item $\mathbf{B}(\mathbf{r})$ is the magnetic field at point $\mathbf{r}$,
  \item $\mu_0$ is the permeability of free space,
  \item $I$ is the current,
  \item $d\mathbf{l}$ is an infinitesimal length element along the current,
  \item $\mathbf{r'}$ is the vector connecting the current element to the point of interest, and
  \item $|\mathbf{r'}|$ is the magnitude of $\mathbf{r'}$.
\end{itemize}

Biot–Savart law is an equation describing the magnetic field generated by a constant electric current. It relates the magnetic field to the magnitude, direction, length, and proximity of the electric current. The Biot–Savart law is fundamental to magnetostatics, playing a role similar to that of Coulomb's law in electrostatics.The law is valid in the magnetostatic approximation, and consistent with both Ampère's circuital law and Gauss's law for magnetism.



The integral is usually around a closed curve, since stationary electric currents can only flow around closed paths when they are bounded. However, the law also applies to infinitely long wires.
To apply the equation, the point in space where the magnetic field is to be calculated is arbitrarily chosen $(\mathbf {r} )$. Holding that point fixed, the line integral over the path of the electric current is calculated to find the total magnetic field at that point. The application of this law implicitly relies on the superposition principle for magnetic fields, i.e. the fact that the magnetic field is a vector sum of the field created by each infinitesimal section of the wire individually.
The Biot–Savart law can be used in the calculation of magnetic responses even at the atomic or molecular level, e.g. chemical shieldings or magnetic susceptibilities, provided that the current density can be obtained from a quantum mechanical calculation or theory.
\begin{itemize}
  \item Name: Padma priya R
  \item Github userid: padmapriyar04
\end{itemize}
\footnote{Wikipedia,"Article:Biot–Savart law",(Accessed: June 9, 2023)}
\end{document}

