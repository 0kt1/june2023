\documentclass[12pt,pdftex]{article}
\usepackage{pslatex}
\usepackage[top=0in,bottom=1in,left=1in,right=1in]{geometry}
\usepackage{float, hyperref}
\usepackage{amssymb, amsmath, amsfonts}
\renewcommand*\ttdefault{txtt}
\renewcommand*\familydefault{\ttdefault}
\usepackage[T1]{fontenc}

\title{ID2090 : Assignment 4}
\author{Aayush Bhakna : CH22B008}
\date{\today}

\begin{document}
	\maketitle
	\thispagestyle{empty}
	\vspace{1cm}
	\section{CH22B008}
	\subsection{Student-Info}
		\textbf{FULL NAME}: Aayush Bhakna \\
		\textbf{GITHUB ID}: Bhakna \\
		\textbf{GITHUB PAGE}: \url{https://github.com/Bhakna} \\
	\subsection{Equation}
		\vspace{0.5cm}
		\begin{align*}
			y_{n+1} = \mu y_n[1 - y_n]
		\end{align*}\\
		The above given equation is known as the \textit{logistic difference equation}\footnotemark. \\\\This equation was initially created to approximately represent the population of non-human species of animals. The $y_n$ is a number between zero and one, which represents the ratio of existing population to the maximum possible population. This nonlinear difference equation is intended to capture two effects -\\ \textbf{reproduction} and \textbf{starvation}. The usual values of interest for the parameter $\mu$ are those in the interval $[0, 4]$, so that $y_n$ remains bounded on $[0, 1]$.\\\\
		It is to be noted that this equation is now used as an example to show the\\ behaviour of chaos and study it's applications and unusual appearances in nature.
		
	\footnotetext{J.O. Adeleke, "Analysis of Logistic Maps" Page-19, doi:\href{https://dx.doi.org/10.13140/RG.2.2.27790.23367}{10.13140/RG.2.2.27790.23367} 
	}
\end{document}