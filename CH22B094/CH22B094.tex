\documentclass{article}
\usepackage{graphicx} % Required for inserting images

\title{Assignment 4 - ID2090}
\author{S Sriram Prashanth ch22b094}
\date{June 2023}

\begin{document}

\maketitle
\begin{center}
    USER ID: Sriram642005
\end{center}

\clearpage
\section{Gram-Schmidt orthogonalization}
Let $v_1,v_2,v_3...v_n \in F^n$. We say that these vectors are orthogonal if and only if $\langle v_i, v_j \rangle = 0$ for all pairs of indices i and j with $ i \neq j$.
\begin{itemize}
    \item If a set of vectors cannot be written in a linear combination until all the coefficients in the combinations are zero, then they are called linearly independent vectors.
    \item Linear independence of a finite list of vectors can be determined using inner product.
    \item Any orthogonal list of non-zero vectors is linearly independent.
\end{itemize}
It will be convenient to use the following terminology. We denote the set of all
linear combinations of vectors $v_1,v_2,v_3...v_n$ by $span(v_1,v_2...v_n)$ 
The procedure of Gram-Schmidt orthogonalization, constructs orthogonal vectors $v_1,v_2,...,v_n$ from the given vectors $u_1,u_2,...,u_n$ so that,
\begin{center}
    span$(v_1,v_2,...,v_n)$ = span$(u_1,u_2,...,u_n)$, $k \leq m$.
\end{center}
The theorm defines $v_1$ = $u_1$ and all subsequent terms of the form $v_m$ are given as:
\begin{equation}
    v_m = u_m - ( \langle u_m,v_1 \rangle / \langle v_1,v_1 \rangle ) v_1 - ... - ( \langle u_m,v_{m-1} \rangle / \langle v_{m-1},v_{m-1} \rangle ) v_{m-1} \cite{MA1102}
\end{equation}
In the above process if $u_i$ is 0, then both $u_i$ and $v_i$ are ignored in the equation.
\bibliographystyle{plain}
\bibliography{References}

\end{document}
