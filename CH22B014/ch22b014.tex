\documentclass[pdftex]{article}
\usepackage{graphicx} % Required for inserting images
\usepackage{hyperref}

\title{Leibniz Integral Rule}
\author{Om Mineeyar GitHub-OmMineeyar}
\date{10 June 2023}

\begin{document}

\maketitle

\section*{CH22B014}
In calculus, the Leibniz integral rule for differentiation under the integral sign states that for an integral of the form
\begin{equation}
	\int_{a(x)}^{b(x)}{f(x,t)dt}
\end{equation}
where $|{a(x),b(x)}|>{\infty}$ and the integrands are functions dependent on $x$, the derivative of this integral is expressible as
\begin{equation}
    {\displaystyle {\frac {d}{dx}}\left(\int _{a}^{b}f(x,t)\,dt\right)=\int _{a}^{b}{\frac {\partial }{\partial x}}f(x,t)\,dt.}
\end{equation}
If $a(x)=a$ is constant and $b(x)=x$, which is another common situation (for example, in the proof of Cauchy's repeated integration formula), the Leibniz integral rule becomes:
\begin{equation}
{{\frac {d}{dx}}\left(\int _{a}^{x}f(x,t)\,dt\right)=f{(x,x)}+\int _{a}^{x}{\frac {\partial }{\partial x}}f(x,t)\,dt}
\end{equation}
\Large{General form: differentiation under the integral sign}
\normalsize
\newline
\newline
Theorem — Let $f(x,t)$ be a function such that both $f(x,t)$ and its partial derivative $f_{x}(x,t)$ are continuous in $t$ and $x$ in some region of the 
$xt$-plane, including $a(x){\le}t{\le}b(x)$, $x_o{\le}x{\le}x_1$. .Also suppose that the functions $a(x)$ and $b(x)$ are both continuous and both have continuous derivatives for  $x_o{\le}x{\le}x_1$. Then for  $x_o{\le}x{\le}x_1$,
\begin{equation}
{{\frac {d}{dx}}\left(\int _{a(x)}^{b(x)}f(x,t)\,dt\right)=f(x,b(x))\,b^{\prime }(x)-f(x,a(x))\,a^{\prime }(x)+\displaystyle \int _{a(x)}^{b(x)}f_{x}(x,t)\,dt.}
\end{equation}
\footnotetext[0]{\href{https://en.wikipedia.org/wiki/Leibniz_integral_rule}{https://en.wikipedia.org/wiki/Leibniz\_integral\_rule}}
\end{document}
